\apendice{Especificación de Requisitos}

\section{Introducción}
En este apartado describiremos los requisitos que ha de cumplir nuestra herramienta en cuanto a funcionalidad.

\section{Objetivos generales}
Los objetivos generales de la herramienta son:
\begin{itemize}
	\item Detección de lineas
	\item Trabajar en perspectiva
	\item Implementar un sistema de guiado
\end{itemize}

\section{Catalogo de requisitos}

\subsection{Requisitos Funcionales}

\begin{itemize}
	
	\item R1. Probar sistema de binarización por luminosidad.
	
	\item R2. Probar sistema de binarización por color.
	
	\item R3. Probar el sistema de resolución de distorsión de perspectiva.
	
	\item R4. Calibrar binarización por color.
	
	\item R5. Calibrar distorsión de perspectiva.
	
	\item R6. Calibrar zona segura del sistema de guiado.
	
	\item R7. Calibrar angulo de giro del sistema de guiado.
	
	\item R8. Ejecutar el flujo principal del programa.
	
	\item R9.Solicitar URL del servidor de vídeo.

\end{itemize}

\subsection{Requisitos no Funcionales}

\begin{itemize}

\item El software ha de conseguir la máxima tasa de Fps, idealmente 50.

\item El software necesita de un servidor de vídeo que le proporcione imágenes mediante una url.

\item El software puede ser extendido, por lo que el código debe de ser modular.

\end{itemize}

\section{Especificación de requisitos}

\subsection{R1. Probar sistema de binarización por luminosidad.}
\begin{description}
	\item [Versión] V1.0
	\item [Autor] Antonio de los Mozos Alonso
	\item [Descripción] Probar el sistema de binarización por luminisidad fuera del flujo principal para ver que materiales o forma de iluminación es la adecuada en el entorno.
	\item [Precondición] Haber obtenido una conexión de vídeo, Requisito R9.
	\item [Secuencia Normal] Paso y Descripción
	
		\begin{enumerate}
			\item El usuario seleccióna la opción en el menú principal.
			
			\item El programa abre una ventana donde se ve la imagen real y la imagen binarizada.			
			
			\item El usuario comprueba que la conexión de vídeo es adecuada, se ve sin cortes.
			\item El usuario prueba sus materiales y condiciones de luminosidad.
			\item Pulsar la tecla S para salir.
		\end{enumerate}
	\item [Postcondición] El usuario consigue ver que materiales y luminosidad funcionan mejor para este tipo de binarización.
	\item [Excepciones] Luminosidad no sea lo suficientemente grande como para distinguir algo en la imagen.
	\item [Importancia] Media.
	\item [Comentarios] Ninguno.
\end{description}

\subsection{R2. Probar sistema de binarización por color.}
\begin{description}
	\item [Versión] V1.0
	\item [Autor] Antonio de los Mozos Alonso
	\item [Descripción] Probar el sistema de binarización por color fuera del flujo principal para ver que materiales o forma de iluminación es la adecuada en el entorno.
	\item [Precondición] Haber obtenido una conexión de vídeo, Requisito R9.
	\item [Secuencia Normal] Paso y Descripción
	
		\begin{enumerate}
			\item El usuario selecciona la opción en el menú principal.
			
			\item El programa abre una ventana donde se ve la imagen real y la imagen binarizada.
			
			\item El usuario comprueba que la conexión de vídeo es adecuada, se ve sin cortes.
			\item El usuario prueba sus materiales y condiciones de luminosidad.
			\item Pulsar la tecla S para salir.
		\end{enumerate}
	\item [Postcondición]  El usuario consigue ver que materiales y luminosidad funcionan mejor para este tipo de binarización.
	\item [Excepciones] Luminosidad no sea lo suficientemente grande como para distinguir algo en la imagen.
	\item [Importancia] Media
	\item [Comentarios] Ninguno
\end{description}


\subsection{R3. Probar el sistema de resolución de distorsión de perspectiva.}
\begin{description}
	\item [Versión] V1.0
	\item [Autor] Antonio de los Mozos Alonso
	\item [Descripción] El usuario necesita ver las fases por las que pasa la imagen hasta que se resuelve la distorsión de perspectiva.
	\item [Precondición] Haber calibrado el sistema de binarización Requisito R4 o Requisito R10. Haber calibrado la distorsión de perspectiva,Requisito R5.
	\item [Secuencia Normal] Paso y Descripción
	
		\begin{enumerate}
			\item El usuario selecciona la opción en el menú.
			\item El programa solicita el tipo de binarización y todas las calibraciones necesarias.
			
			\item EL programa abre una ventana donde se ven las fases por las que pasa la imagen.
			
			\item El usuario realiza pruebas en el entorno para ver que materiales, ángulo de la cámara y luminosidad le conviene. 
			\item El usuario sale de la ventana haciendo click en el aspa roja.
		\end{enumerate}
	\item [Postcondición] Permite que el usuario vea las fases por las que pasa la imagen, y permite realizar ajustes en el entorno según le convenga.
	\item [Excepciones] No haber calibrado bien los sistemas necesarios.
	\item [Importancia] Media.
	\item [Comentarios] Ninguno.
\end{description}

\subsection{R4. Calibrar binarización por color.}
\begin{description}
	\item [Versión] V1.0
	\item [Autor] Antonio de los Mozos Alonso
	\item [Descripción] Calibrar el sistema de binarización por color para usarlo en el flujo del programa principal.
	\item [Precondición] Haber realizado las pruebas pertinentes para comprobar que el algoritmo va a funcionar bien en el entorno. R2.
	\item [Secuencia Normal] Paso y Descripción
		\begin{enumerate}
			\item Cuando se requiera realizar la calibración se abrirá una ventana de OpenCV, con la imagen en color y la imagen binarizada.
			
			\item Pinchar sobre el color por el que queramos binarizar.
			
			\item Pulsar la tecla S para salir.
		\end{enumerate}
	\item [Postcondición] El sistema de binarización por color quedará calibrado para ser usado en el flujo principal.
	\item [Excepciones] Luminosidad no sea lo suficientemente grande como para distinguir algo en la imagen.
	\item [Importancia] Alta.
	\item [Comentarios] Ninguno.
\end{description}

\subsection{R5. Calibrar distorsión de perspectiva.}
\begin{description}
	\item [Versión] V1.0
	\item [Autor] Antonio de los Mozos Alonso
	\item [Descripción] Permite calibrar el sistema de resolución de distorsión por perspectiva.
	\item [Precondición] Necesitamos tener una plantilla cuadrada, y que el sistema de binarización la detecte adecuadamente.
	\item [Secuencia Normal] Paso y Descripción
		\begin{enumerate}
			\item El sistema pedirá que se calibre la distorsión de perspectiva.
			\item El sistema solicitará que se coloque la plantilla del cuadrado delante de la imagen.
			\item El sistema abrirá una ventana OpenCV donde se verá la imagen binarizada.
			\item El usuario tendrá que colocar el cuadrado en la parte baja de la imagen, lo más centrado y paralelo a la cámara posible.
			\item El usuario pulsara S para salir de la ventana.
			\item El sistema pedirá al usuario que retire la plantilla del cuadrado.
			\item El sistema mostrará en la linea de comandos el coeficiente calculado asociado al ángulo.
			
			\item Se calibrará el sistema de resolución de distorsión de perspectiva.
		\end{enumerate}
	\item [Postcondición] El sistema de resolución de distorsión de perspectiva quedará calibrado.
	\item [Excepciones] Colocar el cuadrado de una forma no adecuada, dando lugar a calibraciones erróneas.
	\item [Importancia] Alta.
	\item [Comentarios] Ninguno.
\end{description}

\subsection{R6. Calibrar zona segura del sistema de guiado.}
\begin{description}
	\item [Versión] V1.0
	\item [Autor] Antonio de los Mozos Alonso
	\item [Descripción] El sistema de guiado cuenta con una zona segura que se tiene que calibrar. 
	\item [Precondición] Haber obtenido un flujo de video. Requisito R9.
	\item [Secuencia Normal] Paso y Descripción
		\begin{enumerate}
			\item El sistema solicitará un valor numérico decimal por linea de comandos, entre 0 y 0.3.
			\item El usuario escribirá el valor que desee y pulsará enter.
			\item El sistema validará el valor y continuará.
		\end{enumerate}
	\item [Postcondición] La zona segura del sistema de guiado se generará.
	\item [Excepciones] 
	\item [Importancia] Alta
	\item [Comentarios] Ninguno.
\end{description}

\subsection{R7. Calibrar angulo de giro del sistema de guiado.}
\begin{description}
	\item [Versión] V1.0
	\item [Autor] Antonio de los Mozos Alonso
	\item [Descripción] Permite establecer el ángulo de giro máximo que puede realizar el vehículo, para calibrar el sistema de guiado para ese vehículo.
	\item [Precondición] Haber establecido la zona segura del sistema de guiado. Requisito R6.
	\item [Secuencia Normal] Paso y Descripción
		\begin{enumerate}
			\item El sistema solicitará un valor numérico decimal por linea de comandos, entre 0 y 90.
			\item El usuario escribirá el valor que desee y pulsará enter.
			\item El sistema validará el valor y continuará.
		\end{enumerate}
	\item [Postcondición] El sistema de guiado quedará calibrado para ese tipo de vehículo.
	\item [Excepciones] Si no se calibra bien, el vehículo puede tener problemas a la hora de realizar giros.
	\item [Importancia] Alta
	\item [Comentarios] Ninguno.
\end{description}

\subsection{R8. Ejecutar el flujo principal del programa.}
\begin{description}
	\item [Versión] V1.0
	\item [Autor] Antonio de los Mozos Alonso
	\item [Descripción] Poder ejecutar el flujo del programa principal con todos los sistemas de guiado y resolución de distorsión por perspectiva.
	\item [Precondición] Haber calibrado bien el sistema de binarizacion, el sistema de resolución de distorsión por perspectiva y el sistema de guiado. Requisito R7, Requisito R6, Requisito R5, Requisito R4 y Requisito R9.
	\item [Secuencia Normal] Paso y Descripción
		\begin{enumerate}
			\item El usuario elegirá la opción en el menú principal.
			\item El sistema pedirá hacer la calibración de binarización por color, si se usa.
			
			\item El sistema pedirá hacer la distorsión de perspectiva.
			\item El sistema pedirá hacer la calibración de la zona segura del sistema de guiado.
			
			\item El sistema pedirá hacer la calibración del angulo de giro del sistema de guiado.
			
			\item El sistema abrirá una ventana de OpenCV donde se mostrará el sistema de guiado y la resolución de distorsión por perspectiva.
		\end{enumerate}
	\item [Postcondición] El sistema comenzará a generar instrucciones de guía.
	\item [Excepciones] Alguna de las calibraciones no se haya hecho de forma adecuada.
	\item [Importancia] Alta.
	\item [Comentarios] Ninguno.
\end{description}

\subsection{R9.Solicitar URL del servidor de vídeo.}
\begin{description}
	\item [Versión] V1.0
	\item [Autor] Antonio de los Mozos Alonso
	\item [Descripción] Al inicio del programa se nos solicitará la URL de donde vamos a obtener el vídeo.
	\item [Precondición] Haber iniciado el programa.
	\item [Secuencia Normal] Paso y Descripción
		\begin{enumerate}
			\item El programa solicitará una URL de video.
			\item El usuario introducirá por linea de comandos la URL.
			\item El programa validará que tiene conexion a la URL.
		\end{enumerate}
	\item [Postcondición] Se establecerá una conexión con el servidor de video.
	\item [Excepciones] No poderse conectar con el servidor de video.
	\item [Importancia] Alta.
	\item [Comentarios] Ninguno.
\end{description}

\section{Casos de uso}
Diagrama de casos de uso de la herramienta:
\imagen{casos}{Diagrama de casos de uso.}


