\apendice{Especificación de diseño}

\section{Introducción}
En este apartado veremos como se ha diseñado la herramienta. 

\section{Diseño de datos}
El único dato relevante con el que trabajamos son imágenes. Veamos los formatos de imagen usados:


\begin{itemize}
	\item Las imágenes se generan en formato RGB, el cual tiene 3 matrices de enteros de 8 bits, una para Rojos, Una para Verdes y otra para Azules.
	
	\item Las imágenes en formato BGR, tienen 3 matrices de enteros de 8 bits, una para Azules, una para Verdes y otra para Rojos.
	
	\item Las imágenes en escala de grises tiene una sola matriz de valores enteros de 8 bits.
	
	\item Las imágenes en formato HSV tienen 3 matrices de enteros de 8 bits, una para los colores, H; otra para la saturación, S; y otra para el valor de negro, V.
\end{itemize}

Las transformaciones entre formatos se harán con llamadas a OpenCV.


\section{Diseño procedimental}

A continuación veremos la interacción entre los elementos del programa.

\imagen{bin_color}{Interacción. Prueba de binarización por color.}

\imagen{bin_lum}{Interacción. Prueba de binarización por luminosidad.}

\imagen{color_persp}{Interacción. Probar perspectiva con binarización por color.}

\imagen{lum_persp}{Interacción. Probar perspectiva con binarización por luminosidad.}

\imagen{gui_color}{Interacción. Ejecución principal con binarización por color.}

\imagen{gui_lum}{Interacción. Ejecución principal con binarización por luminosidad.}



\section{Diseño arquitectónico}
A continuación se mostrará la organización de los diferentes módulos del programa.

Destacar que en el diseño del programa se han usado clases y scripts, remarcados con la palabra (Script) al lado de su nombre en el diagrama.

\imagen{class}{Diagrama de clases y scripts. }


