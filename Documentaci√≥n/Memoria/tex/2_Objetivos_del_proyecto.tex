\capitulo{2}{Objetivos del proyecto}

\section{Objetivos generales}
Desarrollar un software capaz de interpretar una imagen, para generar un conjunto de instrucciones.

\begin{itemize}
	\item Distinguir en esa imagen una línea guía.
	
	\item Obtener la trayectoria que sigue esa guía.
	
	\item Generar una serie de instrucciones capaces de dirigir a un vehículo de guiado autónomo (AGV).
\end{itemize} 

\section{Objetivos funcionales}
\begin{itemize}
	\item El usuario podrá elegir el tipo de binarización, por luminosidad o por color.
	
	\item El usuario podrá colocar la cámara en diferente ángulo respecto del suelo, dentro de los margenes de un mínimo de 30 grados y un máximo de 70.
	
	\item El usuario podrá usar una plantilla para corregir la distorsión por perspectiva.
	
	\item El usuario podrá probar los distintos tipos de binarización fuera del flujo del programa principal, en vista de realizar ajustes y pruebas.
	
	\item El usuario podrá ver las distintas fases por las que pasa la imagen, desde que es capturada hasta que se detecta la trayectoria, en vista de realizar ajustes y pruebas.
	
	\item El usuario podrá ver las instrucciones de guiado, situación de luminosidad, fotogramas por segundo y situación de trayectoria en el flujo principal del programa.
	
\end{itemize}


\section{Objetivos tecnológicos}
\begin{itemize}
	\item Obtener un flujo de imagen continuo, por lo que necesitaremos un hardware capaz de transmitir  de forma continua.
	
	\item Obtener un procesamiento de imagen de más de 50 fotogramas por segundo, ya que el AGV requiere de una instrucción cada 2 centésimas de segundo.
	
	\item Desarrollo con interfaz en linea de comandos, para hacer la aplicación lo más ligera posible.
	
	\item Generar un conjunto de pruebas de todas las funciones creadas y/o probadas.
	
	\item Controlar la luminosidad en la imagen para poder reproducir el conjunto de pruebas.
	
	\item Utilización de OpenCV para el procesamiento de la imágen.
	
	\item Utilización de SciPy para generar funciones mediante regresión lineal.
	
	\item Utilización de MatPlotLib para crear ventanas donde ver comparaciones de imágenes.
\end{itemize}
