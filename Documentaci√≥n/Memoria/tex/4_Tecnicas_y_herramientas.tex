\capitulo{4}{Técnicas y herramientas}

\section{Técnicas de desarrollo}

\subsection{Metodología ágil}
Por ser un proyecto software, las ventajas de usar una metodología ágil\cite{met_agil} son claras:

\begin{itemize}
	\item Las personas que participan en el desarrollo adquieren roles diferenciados.
	\item El desarrollo es incremental, por etapas. Al final de cada etapa se entrega una parte del software funcional al cliente.
	\item Partimos de unos requisitos básicos y podemos ir añadiendo más a lo largo del desarrollo, o modificar alguno de los existentes.
	\item La comunicación con el cliente se hace a lo largo de todo el desarrollo, favoreciendo que el producto final sea lo que esperaba.
	\item Favorecer la comunicación del equipo en lugar de la excesiva documentación.
	\item Buscar la calidad del resultado en los conocimientos de las personas que han participado en el desarrollo, y no en los procesos empleados para hacerlo.
	\item Solapar las fases de desarrollo en lugar de hacerlas de forma secuencial o en cascada.
\end{itemize}

\subsection{Scrum}
Scrum\cite{scrum} es un marco de trabajo dentro de las metodologías ágiles, veamos los 3 roles de esta forma de trabajo:

\begin{itemize}
	\item Product Owner: es el cliente. Propietario de la idea o proyecto a desarrollar. 
	\item ScrumMaster: experto en Scrum, ayuda tanto al Product Owner como al Equipo Scrum a alcanzar los objetivos finales del proyecto.
	\item Equipo Scrum: equipo de personas que va a realizar el proyecto. Es conveniente que sea un equipo multidisciplinario, en el que cada persona sea experta o conocedora de un campo diferente a las demás. El equipo debe de ser de 5 a 9 personas, para favorecer la comunicación y fluidez del desarrollo.
\end{itemize}

Para este proyecto en concreto, el alumno tomara el rol de Equipo Scrum y ScrumManager. El profesor actuará como cliente.

Este proyecto software está más enfocado a investigación, realizar pruebas y obtener resultados, por lo que partiremos desde un requisito inicial básico, y cuando el cliente lo consideré valido y completado, propondrá mas requisitos. 

Esto, aunque no contradice la forma de trabajar de las metodologías ágiles, otorga una perspectiva diferente, ya que no tenemos determinadas unas etapas concretas del desarrollo. Realizaremos etapas más o menos cortas en función de lo que dure la investigación del nuevo requisito y la implementación y pruebas que lo resuelvan.

En cada sprint se explorarán o implementarán diferentes herramientas en busca de un objetivo, se harán pruebas funcionales con alguna de esas herramientas, estableciendo los parámetros iniciales de cada prueba, el resultado esperado y obtenido. El software evolucionará con cada prueba.

\section{Herramientas de documentación}

\subsection{LaTeX}

Latex es un sistema de composición de texto, que busca la creación de documentos con una alta calidad tipográfica.

La documentacion de este proyecto se ha creado usando TexMaker, software gratuito bajo la licencia GPL.

Página oficial editor Latex: \url{http://www.xm1math.net/texmaker/}

También se ha usado MikTex como implementación de Latex para windows. En su página dicen que por ser un conjunto de paquetes, no tienen una licencia concreta como en otros casos, aun así siguen las directrices de la FSF\cite{fsf}, y establecen una serie de pautas de uso distribuición y modificación.

Página oficial MikTex: \url{https://miktex.org/}

\section{Herramientas de gestión}

\subsection{Git}
Git es un sistema de control de versiones, actualmente el más utilizado. Está integrado en diferentes plataformas para albergar proyectos como por ejemplo: GitHub, GitLab y Bitbucket.
Ha sido elegido por ser el que hemos usado durante toda la carrera, lo cual facilitara la utilización.

Git tiene licencia de software libre GNU LGPL v2.1.

Página oficial de Git: \url{https://git-scm.com/}

\subsection{GitHub}
GitHub es una plataforma online para albergar proyectos desarrollados mediante un sistema de control de versiones git. 

Ha sido elegido frente a otras alternativas como GitLab o Bitbucket, por la simplicidad de organización de su página web, y por la integración con extensiones como ZenHub.

Es gratuita para proyectos open source.

Página oficial de GitHub: \url{https://github.com/}

Es una herramienta mundialmente conocida en el mundo del desarrollo de software, y una de las más usadas para este tipo de propósitos.

Principalmente tiene 3 tipos de elementos:
\begin{itemize}
	\item Milestones: Son puntos temporales de importancia dentro de la planificación del proyecto.
	\item Issues: Tareas a realizar. Pueden ser asignadas a um Milestone. 
 	\item Commits: Son el elemento que permite ir realizando cambios en el repositorio: añadir elementos, modificar, eliminar... Sirven para ir resolviendo las tareas que vayan surgiendo.
\end{itemize}

La usaremos para albergar el código del proyecto, y para aplicar metodologías ágiles y scrum. Los Sprints los representaremos con Milestones. Las tareas que vayan surgiendo en cada Sprint como Issues, y la resolución de las diferentes tareas se harán con commits. He elegido esta herramienta por haberla usado en alguna asignatura en la carrera y estar familiarizado con ella, y por ser uno de los repositorios de git más usados.

\subsection{Zenhub}
Zenhub es una extension para Google Chrome. Extiende las posibilidades de GitHub, añadiendo elementos como tableros para organizar las tareas, nuevas medidas para evaluar el desarrollo del proyecto y gráficas asociadas a estas medidas. También podemos gestionar el backlog mediante los tableros.

He elegido esta herramienta por ser un gran añadido a GitHub, y por haberla usado en alguna asignatura de la carrera.  

Es gratuita para proyectos open source.

Página ofocial de ZenHub: \url{https://www.zenhub.com/}


\section{Herramientas de desarrollo}

\subsection{Python}
El lenguaje de programación usado para este proyecto es python. Ha sido elegido, primero porque las librerías de OpenCV están para este lenguaje, lo cual es un requisito imprescindible; y segundo por la facilidad de uso, la versatilidad y la gran cantidad de librerías externas con las que cuenta.

Concretamente se ha usado python en su versión 2.7.5.
Nos hubiera gustado dar el salto a python 3, pero no hay librerías de OpenCV de forma oficial para esta versión, solo librerías traducidas de python 2 a python 3, por usuarios, por lo que no contamos con un soporte oficial y nos podemos encontrar con errores y sorpresas desagradables.

Tiene licencia Open Source compatible con GPL.

Página oficial de Python: \url{https://www.python.org/}

\subsection{PyLint}
PyLint es una herramienta para analizar y encontrar errores en códigos python. Se puede añadir a IDEs de forma sencilla, y se pueden personalizar los distintos avisos que proporciona. 
Además permite refactorizar código y cuenta la guía de estilos pep8\cite{pep8} para estructurar adecuadamente y mejorar la interpretación del código.

Ha sido elegido como ayuda a la hora de programar, por su integración con el IDE Visual Studio Code.

Página oficial de PyLint: \url{https://www.pylint.org/}

\subsection{OpenCV}
OpenCV es una librería software de vision aritificial y machine learning. Nos proporcionará todos los elementos software necesarios para poder obtener y procesar imágenes. 
OpenCV se distribuye bajo licencia BSD. 
Esta librería está implementada para varios lenguajes de programación, aunque la elección fue python por la facilidad de uso, la ligereza del lenguaje, y las posibilidades de extension que ofrece.

Ha sido elegida por ser la librería por excelencia para realizar proyectos de procesamiento de imágenes y visión artificial.

Página oficial de la librería OpenCV: \url{https://opencv.org/}

\subsection{SciPy}
SciPy es una librería escrita en Python, Fortran, C y C++, para Python destinada a la computación técnica y científica. 
Su licencia es BSD-new license.
Usaremos la versión más actual, 1.1.0.

Página oficial de la librería SciPy: \url{https://www.scipy.org/}

\subsection{NumPy}
NumPy es una librería esencial para trabajar con arrays y matrices en Python. Incluye una gran cantidad de funciones matemáticas para aplicar a este tipo de estructuras de información.
Tiene licencia BSD-new license
Usaremos la versión 1.13.1.

Página oficial de la liberría NumPy: \url{http://www.numpy.org/}

\subsection{MatPlotLib}
MatPlotLib es una librería de representación matemática, aunque también es usada de forma general para gestionar ventanas de interfaz gráfica.
Tiene licencia BSD-new license
Usaremos la version 1.3.0.

Página oficial de la librería MatPlotLib: \url{https://matplotlib.org/}

\subsection{Urllib}
Urllib es una librería para gestionar conexiones con urls. La usaremos para conectar con la url del servidor de vídeo. 
Tiene licencia MIT.
Usaremos la versión 1.17.

\subsection{Time}
Time es una librería propia de python que permite gestionar eventos temporales. La usaremos para registrar los tiempos de procesado de imagen, y calcular los fotogramas por segundo. 


\subsection{Visual Studio Code}
Visual Studio Coce es el IDE elegido para este proyecto. Es un IDE sencillo, compatible con una gran variedad de lenguajes de programación y compatible con Linux, Windows y MacOS.
 
Se pueden añadir extensiones de una forma muy sencilla dentro del propio IDE, de hecho es el propio IDE el que nos sugiere instalar algunas extensiones dependiendo del lenguaje de programación que estemos usando. Además la comunidad puede desarrollar extensiones para el IDE, lo que amplia aún más las posibilidades de personalización.

Además de todo esto, es compatible con git de una forma muy sencilla, simplemente abrimos la carpeta de nuestro repositorio y ya entra en funcionamiento el control de versiones.

Por todo esto, considero que esta herramienta es la más adecuada y la que mejor se adapta a mis requisitos para realizar este proyecto.

Su propietario es Microsoft y sus licencias son:
\begin{itemize}
	\item Codigo fuente, licencia del MIT.
	\item Binarios, licencia FreeWare.
\end{itemize}

Página oficial de Visual Studio Code: \url{https://code.visualstudio.com/}

\subsection{IPWebcam}
Aplicación android, con dos versiones, gratuita y de pago. En este proyecto usaremos la versión gratuita.
Su funcionalidad es crear un streaming local de vídeo en directo, accesible desde una url. Es útil puesto que me permite tener la cámara totalmente libre, sin ataduras, pudiéndola mover como desee.

Descarga de IPWebcam desde Play Store: \url{https://play.google.com/store/apps/details?id=com.pas.webcam&hl=en}

Ha sido elegida por su simplicidad de funcionamiento, y por poder realizar ajustes desde una web accesible desde la url generada por la aplicación.

Se exploró otra opción, el usar la cámara integrada del portatil, pero el smartphone tiene unas ventajas muy claras:

\begin{itemize}
	
	\item Una ventaja clara es que no es lo mismo mover un smartphone a mover el portátil, y de cara a realizar pruebas del programa, vamos a tener que mover la cámara en la mayoría de ocasiones, por lo que contra más pequeña sea la cámara o el dispositivo en el que esta se encuentra, mejor.
	
	\item La cámara del smartphone es muy superior a la cámara web del portatil, en cuanto a resolución, tasa de frames, apertura del objetivo... La posibilidad de personalización de la imagen, y la calidad final de la imagen es muy superior a la que nos puede ofrecer la webcam del portatil.
	
\end{itemize}

Por estos motivos se ha optado por la opción de usar el smartphone como cámara, y es por esto por lo que nos vemos obligados a usar una aplicación capaz de retransmitir vídeo en directo via inalambrica, en este caso la aplicación IPWebcam.

\subsection{Online pixel ruler}
Online pixel ruler es una herramienta online que permite realizar medidas en imágenes. Se ha optado por esta herramienta por su simplicidad, y su disponibilidad sin necesidad de instalar ningun software. 

Se puede acceder a ella desde la url: \url{https://www.rapidtables.com/web/tools/pixel-ruler.html}


\subsection{Soporte para la cámara}
Para permitir el movimiento de la cámara desde la posición que se requiere para realizar este proyecto, se ha creado un soporte con piezas de Lego con ruedas, lo cual nos permite mover la cámara adelante y atrás del recorrido.
 
\subsection{Plantillas}
Se han creado diferentes plantillas para representar líneas, ángulos y cuadrados, todos ellos necesarios en las fases de desarrollo o pruebas del proyecto. Las plantillas están creadas a papel, y remarcadas con cinta aislante.

\imagen{plantillas}{Plantillas usadas en el proyecto.}
