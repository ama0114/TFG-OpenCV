\capitulo{3}{Conceptos teóricos}

\section{Implementación hardware}
En este proyecto trabajaremos con AGVs. Necesitaremos 3 elementos principales para implementarlo:

\begin{itemize}
	\item Una cámara que capte imágenes, situada adecuadamente en el AGV.
	
	\item Una ordenador que ejecute el programa.
	
	\item Un elemento para transmitir las instrucciones que genere nuestro programa al sistema de control del vehiculo.
	
\end{itemize}

Tras analizar los diferentes elementos y la forma de combinarlos, podemos obtener dos formas de organización de estos elementos, veamos ambas, y sus ventajas e inconvenienetes.
 

\subsection{Implementación total en el AGV}
En esta implementación se colocarán todos los elementos en el AGV, ejecutando el programa en el ordenador del propio vehículo.
\subsubsection{Ventajas}
\begin{itemize}

	\item No se necesita ningún elemento extra.
	
	\item La transmisión de la imagen y de las instrucciones es inmediata y por conexión física, lo que asegura una buena transmisión de la información.
	
\end{itemize} 

\subsubsection{Desventajas} 
\begin{itemize}

	\item La capacidad de procesamiento del ordenador del vehiculo no es muy grande.
	
	\item Esta implementación dificulta el control simultaneo de varios AGVs.
	
\end{itemize}

\subsection{Externalización del ordenador que ejecuta el programa} 

En esta implementación se externaliza el ordenador donde se ejecuta el programa, a un ordenador fijo, más potente.
\subsubsection{Ventajas}
\begin{itemize}
	\item El ordenador externo nos ofrece mayor capacidad de procesamiento.
	
	\item Poder ver las instrucciones y situación en tiempo real que se están mandando.
	
\end{itemize}

\subsubsection{Desventajas}
\begin{itemize}
	\item Se necesita hardware extra, además del que lleva el AGV.
	
	\item Necesitamos hardware que nos permita un buen flujo de imágenes hacia el ordenador, y una buena transmisión de las instrucciones hacia el vehículo.
	
\end{itemize} 

\section{Referencias}

Las referencias se incluyen en el texto usando cite \cite{wiki:latex}. Para citar webs, artículos o libros \cite{koza92}.


\section{Imágenes}

Se pueden incluir imágenes con los comandos standard de \LaTeX, pero esta plantilla dispone de comandos propios como por ejemplo el siguiente:

\imagen{escudoInfor}{Autómata para una expresión vacía}



\section{Listas de items}

Existen tres posibilidades:

\begin{itemize}
	\item primer item.
	\item segundo item.
\end{itemize}

\begin{enumerate}
	\item primer item.
	\item segundo item.
\end{enumerate}

\begin{description}
	\item[Primer item] más información sobre el primer item.
	\item[Segundo item] más información sobre el segundo item.
\end{description}
	
\begin{itemize}
\item 
\end{itemize}

\section{Tablas}

Igualmente se pueden usar los comandos específicos de \LaTeX o bien usar alguno de los comandos de la plantilla.

\tablaSmall{Herramientas y tecnologías utilizadas en cada parte del proyecto}{l c c c c}{herramientasportipodeuso}
{ \multicolumn{1}{l}{Herramientas} & App AngularJS & API REST & BD & Memoria \\}{ 
HTML5 & X & & &\\
CSS3 & X & & &\\
BOOTSTRAP & X & & &\\
JavaScript & X & & &\\
AngularJS & X & & &\\
Bower & X & & &\\
PHP & & X & &\\
Karma + Jasmine & X & & &\\
Slim framework & & X & &\\
Idiorm & & X & &\\
Composer & & X & &\\
JSON & X & X & &\\
PhpStorm & X & X & &\\
MySQL & & & X &\\
PhpMyAdmin & & & X &\\
Git + BitBucket & X & X & X & X\\
Mik\TeX{} & & & & X\\
\TeX{}Maker & & & & X\\
Astah & & & & X\\
Balsamiq Mockups & X & & &\\
VersionOne & X & X & X & X\\
} 
