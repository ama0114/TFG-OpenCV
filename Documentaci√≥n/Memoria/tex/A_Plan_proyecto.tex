\apendice{Plan de Proyecto Software}

\section{Introducción}
\cite{koza92}
En esta sección se detallará la planificación del proyecto, y la viabilidad del mismo. 

Como planificación se entienden las distintas etapas por las que va a pasar el proyecto hasta su finalización, y la duración de cada etapa.

La viabilidad la podemos definir como la posibilidad de realizar el proyecto.

En este proyecto no ha habido una planificación demasiado estricta en cuanto a tiempos de entrega, ya que es un proyecto con una gran componente de investigación. Aún así, se han seguido las directrices de las metodologías ágiles, en concreto dentro del marco de Scrum\cite{scrum}.

La viabilidad económica de una investigación puede llegar a ser difícil de estimar, ya que se intenta agotar los recursos disponibles al máximo, tanto temporalmente como económicamente. Aun así, llegado al punto temporal al que se ha llegado, podemos analizar la viabilidad económica en retrospectiva.

\section{Planificación temporal}
Inicialmente planteamos usar una metodología mas acorde a una investigación, basada en la exploración(etapa de investigación) y evolución(en base a realizar pruebas). Como el producto final ha de ser un software que recoja los resultados de todas las investigaciones y pruebas, finalmente buscamos una metodología ágil, Scrum. Aun así, dentro de los sprints e iteraciones, es donde se han hecho las etapas de exploración y evolución.

\subsection{Sprint 0: 1/10/17-9/11/17}
El proyecto parte de un sprint inicial, en el que se explicó el contexto y los elementos con los que se tenía que trabajar. Se preparó del entorno de trabajo, con el hardware y software necesario. Se buscó la mejor forma para realizar el proyecto, explorando dos opciones, trabajar con videos grabados, o trabajar conun streaming en directo.

\subsection{Sprint 1: 9/11/17-23/11/17}
En el primer sprint realizado se buscó la forma de obtener la tasa de Fps necesaria para cumplir con los requisitos establecidos, así como realizar una prueba de binarización de la línea guía, hecha de unos materiales concretos, buscar diferentes algoritmos, probarlos, generar una funcionalidad para medir la luminosidad y así establecer los parametros bajo los que se realizan las pruebas.

\subsection{Sprint 2: 23/11/17-14/12/17}
En el segundo sprint buscamos el tomar medidas de la línea sobre la imagen, para comprobar la eficacia de los metodos de binarizacion, se probaron diferentes funciones que median el ancho real y el ancho calculado de la línea.pip i

\subsection{Sprint 3: 12/2/18-30/04/18}
En el tercer sprint buscamos tomar medidas de una forma más avanzada, no solo si la línea esta de forma perpendicular a la cámara, sino estando en cualquier posición. Para esto necesitaremos entrar en la problematica de la perspectiva. Además iniciaremos la investigacon del sistema de guiado a traves de la trayectoria.

\subsection{Sprint 4: 30/04/18-28/06/18}
En el cuarto sprint buscamos nuevas formas de binarizacion, en concreto la binarizacion por color. Crear una interfaz para dar una funcionalidad central al proyecto, y finalizar el sistema de guiado y su forma de representación. Además se planteó la posibilidad de integrar las librerías de canKin para conectar el sistema de guiado básico con un robot, si la evolución temporal lo permitía.

\section{Estudio de viabilidad}

\subsection{Viabilidad económica}

\subsection{Viabilidad legal}


