\apendice{Plan de Proyecto Software}

\section{Introducción}
En esta sección se detallará la planificación del proyecto. 

Como planificación se entienden las distintas etapas por las que va a pasar el proyecto hasta su finalización, y la duración de cada etapa. También tenemos que tener en cuenta la viabilidad del proyecto, la posibilidad de que el proyecto sea realizado.

En este proyecto no ha habido una planificación demasiado estricta en cuanto a tiempos de entrega, ya que es un proyecto con una gran componente de investigación. Aún así, se han seguido las directrices de las metodologías ágiles, en concreto dentro del marco de Scrum\cite{scrum}.

La viabilidad económica de una investigación puede llegar a ser difícil de estimar, ya que se intenta agotar los recursos disponibles al máximo, tanto temporalmente como económicamente. Aun así, llegado al punto temporal al que se ha llegado, podemos analizar la viabilidad económica en retrospectiva.

\section{Planificación temporal}

Inicialmente planteamos usar una metodología mas acorde a una investigación, basada en la exploración(etapa de investigación) y evolución(en base a realizar pruebas). Como el producto final ha de ser un software que recoja los resultados de todas las investigaciones y pruebas, finalmente aplicamos una metodología ágil, Scrum. Aun así, dentro de los sprints e iteraciones, es donde se han hecho las etapas de exploración y evolución. 

El alumno tomará el papel de equipo de desarrollo y de Scrum Master. El tutor hará de cliente.

Los resultados de las diferentes pruebas realizadas se han ido registrando en el diario de pruebas\cite{diario}.

Cabe decir que a pesar de que el proyecto estaba planeado para iniciarse en el segundo cuatrimestre, hemos empezado en el primero por motivos académicos del alumno, para poder tener más tiempo de desarrollo.

\subsection{Sprint 0: 1/10/17-9/11/17}
El proyecto parte de un sprint inicial, en el que se explicó el contexto y los elementos con los que se tenía que trabajar. Se preparó del entorno de trabajo, con el hardware y software necesario. Se buscó la mejor forma para realizar el proyecto, explorando dos opciones, trabajar con videos grabados, o trabajar conun streaming en directo.

\subsubsection{Tareas}
\begin{itemize}
	\item Preparación de las librerías Python
	\item Preparación del IDE
	\item Preparación del Repositorio
	\item Preparación cámara y aplicación IPWebcam
	\item Preparación soporte cámara 
\end{itemize}
\subsubsection{Backlog}
En este sprint lo que hemos hecho es la fase de preparación del proyecto, preparar todos los materiales. También se dieron los conocimientos básicos al alumno en reuniones con el tutor, sobre el contexto de trabajo, que se pretendía realizar, que se pretendía probar y que se pretendía conseguir. 

\subsubsection{Investicación}
En esta fase se investigaron las formas de organizar los elemtnos con los que se iba a trabajar. Inicialmente se propuso usar la WebCam integrada en el portátil, pero finalmente se optó por usar un smartphone que retransmitiera vídeo, ya que nos da mucha mayor libertad. 

Esto podría asociarse a las formas de organización de los diferentes elementos (cámara, ordenador y sistema de control del AGV) vistas en la memoria. Resaltar que para este proyecto solo hemos trabajado con la cámara y el ordenador, puesto que esta enfocado a la detección de líneas, no al control de un AGV.

\subsubsection{Pruebas}

Se realizaron las pruebas de conexión pertinentes. Inicialmente se usaba el router de la red local de casa, con el smartphone y el ordenador conectados vía wifi. Al ser la conexión inalámbrica, y el router estar recibiendo peticiones de otros dispositivos, había más latencia y perdida de paquetes, lo que generaba un mal rendimiento de la transmisión de vídeo.

A si que finalmente, se optó por usar un router especifico para este sistema, conectando por cable el ordenador para optimizar al máximo la conexión y evitar la perdida de paquetes, y conectado vía wifi al smartphone.


\subsection{Sprint 1: 9/11/17-23/11/17}
En el primer sprint realizado se buscó la forma de obtener la tasa de Fps necesaria para cumplir con los requisitos establecidos, así como realizar una prueba de binarización de la línea guía, hecha de unos materiales concretos, buscar diferentes algoritmos, probarlos, generar una funcionalidad para medir la luminosidad y así establecer los parametros bajo los que se realizan las pruebas.

\subsubsection{Tareas}

\begin{itemize}
	\item Obtener una tasa de Fps que permitiera realizar el procesado de imágenes en menos de 2 centesimas de segundo.
	
	\item Crear la plantilla de fondo blanco y línea negra para hacer la binarización por luminosidad.
	
	\item Buscar diferentes algoritmos de binarización por luminosidad.
	
	\item Probar los diferentes algoritmos, dejar registro de las pruebas, y elegir el que mejor funcione.
	
	\item Generar una funcionalidad para medir la luminosidad en la imagen.
	
	\item Crear un archivo donde ir implementando todas las funciones probadas, con los parámetros necesarios.

\end{itemize}

\subsubsection{Backlog}
En este sprint hemos realizado los ajustes pertinentes en la cámara para cumplir con las restricciónes de Fps, hemos buscado, probado y registrado diferentes algoritmos de binarización, y finalmente hemos dejado una implementación con el mejor algoritmo encontrado. Además para realizar las pruebas se ha necesitado una funcionalidad que permita medir la luminosidad de la imagen, por lo que también se ha implementado esta funcionalidad.

\subsubsection{Investicación}
Se han investigado los diferentes algoritmos de binarización por luminosidad.

\subsubsection{Pruebas}
Se han realizado pruebas con todos los algoritmos, estableciendo parámetros y dejando registros de los resultados.

\subsection{Sprint 2: 23/11/17-14/12/17}
En el segundo sprint buscamos el tomar medidas de la línea sobre la imagen, para comprobar la eficacia de los metodos de binarizacion, se probaron diferentes funciones que median el ancho real y el ancho calculado de la línea, estando la línea en perpendicular a la cámara.

\subsubsection{Tareas}
\begin{itemize}

	\item Diseñar función que permita realizar medidas de la línea sobre la imagen.
	
	\item Diseñar función para estimar el ancho de la línea en perspectiva.
	
	\item Probar las funciones, y comprobar el fallo en la medición de la anchura en diferentes condiciones de luminosidad.
	
\end{itemize}

\subsubsection{Backlog}
Se implementaron las funciones necesarias para medir la línea normal y en perspectiva, estando perpendicular a la cámara. Aunque estas funciones son correctas, la funcionalidad para resolver la distorsión de perspectiva es muy limitada, funcionando solo cuando la línea esta en perpendicular a la cámara.

\subsubsection{Investicación}
Primeros inicios en la resolución de la distorsión de perspectiva, buscando el saber el ancho real de la línea en diferentes zonas de la imagen.

\subsubsection{Pruebas}
Se realizaron pruebas de medición de la línea y comparación con la anchura real calculada.


\subsection{Sprint 3: 12/2/18-30/04/18}

En el tercer sprint buscamos tomar medidas de una forma más avanzada, no solo si la línea esta de forma perpendicular a la cámara, sino estando en cualquier posición. Para esto necesitaremos entrar en la problemática de la perspectiva. Además iniciaremos la investigación del sistema de guiado a través de la trayectoria.
\subsubsection{Tareas}
\begin{itemize}

	\item Avanzar en la resolución de distorsión por perspectiva
	
	\item Implementar una funcionalidad que permita obtener la vista de pájaro.
	
	\item Comprobar que la imagen en vista de pajaro guarda las mismas proporciones que la realidad.
	
	\item Implementar una funcionalidad que permita obtener la trayectoria (centro de la línea).
	
	\item Indagar en los sistemas de guiado, realizando una funcionalidad que calcule la distancia de la trayectoria al centro de la imagen.

\end{itemize}
\subsubsection{Backlog}
En este sprint se hacen avances muy importantes, sobre todo la resolución total de la distorsión por perspectiva, problemática principal del proyecto. Se consigue generar una funcionalidad que genera la vista de pájaro guardando las proporciones reales. Se obtiene la trayectoria mediante los bordes de la línea. Iniciamos el desarrollo del sistema de guiado.
\subsubsection{Investicación}
Se investigan formas de resolver la distorsión por perspectiva adecuadamente, así como de relacionar la funcionalidad que ya tenemos con esta problemática.
Se investigan los diferentes tipos de sistema de guiado.
\subsubsection{Pruebas}
Las pruebas realizadas en el ámbito de la distorsión por perspectiva consisten principalmente en enganchar las diferentes funcionalidades que tenemos y que funcionen en conjunto, y comprobar que la vista de pajaro guarda las proporciones de la realidad, para esto medimos ángulos.
Dentro de los sistemas de guiado, hacemos una prueba que nos permite ver como actúa el sistema de guiado implementado.


\subsection{Sprint 4: 30/04/18-28/06/18}
En el cuarto sprint buscamos nuevas formas de binarizacion, en concreto la binarizacion por color. Crear una interfaz para dar una funcionalidad central al proyecto, y finalizar el sistema de guiado y su forma de representación. Además se planteó la posibilidad de integrar las librerías de canKin para conectar el sistema de guiado básico con un robot, si la evolución temporal lo permitía.

\subsubsection{Tareas}
\begin{itemize}
	\item Indagar en la binarización por color.
	
	\item Implementar un sistema de binarización por color compatible con el resto de funcionalidad.
	
	\item Implementar una interfaz para dar una cohesión a todas las funcionalidades implementadas.
	
	\item Mostrar en la interfaz toda la información de los algoritmos.
	
	\item Integrar librerías canKin, opcional, en función de la evolución del resto de tareas.
	
	\item Generar ejecutable de la aplicación para hacer el release\cite{release}.

\end{itemize}

\subsubsection{Backlog}
Se cumplieron todas las tareas obligatorias, tanto implementar un sistema de binarización por color como la interfaz que conecta todas las funcionalidades implementadas. Además se consiguió mostrar la información en las propias ventanas de OpenCV sin sacrificar rendimiento de la aplicación. Se hizo el release de la aplicación en el repositorio.

\subsubsection{Investicación}
Se investigó acerca de los algoritmos de binarización por color, dando lugar a la implementación de un sistema de binarización por color.

\subsubsection{Pruebas}
Las pruebas realizadas parten desde los distintos materiales que se podían usar para la binarización por color. Inicialmente se uso cartulina amarilla, no funcionaba adecuadamente hasta que remarcamos su borde con cinta aislante negra, lo cual resaltó el borde e hizo que funcionara bien.


\section{Estudio de viabilidad}
Necesitamos que el proyecto sea tanto viable económicamente, es decir, que sea rentable; y que sea viable legalmente, es decir, que no viole alguna ley a lo largo del transcurso del mismo. 

\subsection{Viabilidad económica}
En este apartado se analizarán los costes del proyecto.

\subsubsection{Costes de personal}
Consideraremos que este proyecto se ha realizado por una persona a lo largo de seis meses: Octubre, Noviembre, Febrero, Marzo, Abril y Mayo. Trabajando 20 días al mes, 4 horas diarias.
El sueldo de un programador sin experiencia actualmente son 10 euros por hora. Haciendo cálculos:
$10 euros/hora * 4 horas/dia = 40 euros/dia$
$40 euros/dia * 20dias/mes = 800 euros/mes$
$800 euros/mes * 6 meses = 4800 euros de salario$ 

\subsubsection{Costes de hardware}
En esta sección calcularemos los costes hardware del proyecto. Incluiremos todos los elementos hardware empleados. Consideraremos la vida útil del hardware en 6 años.
La amortización la calcularemos como:

$Amortización = (Valor del bien)/Vida util$


\begin{itemize}
	\item Ordenador portátil: 500 euros.
	
	$500 euros/(12*6)meses = 6.94 euros/mes$
	
	Por tanto en 6 meses:
	
	$6,94 euros/mes * 6 meses = 41.64 euros brutos$
	
	\item Smartphone: 300 euros.
	$300 euros/(12*6)meses = 4.16 euros/mes$
	
	Por tanto en 6 meses:
	
	$4.16 euros/mes * 6 meses = 24.96 euros brutos$
	
	\item Router: 50 euros.
	$50 euros/(12*6)meses = 0.69 euros/mes$
	
	Por tanto en 6 meses:
	
	$0.69 euros/mes * 6 meses = 4.14 euros brutos$
	
\end{itemize}

\subsubsection{Costes de software}
La única licencia no gratuita usada ha sido el sistema operativo Windows 10 Home, que tiene un coste de 120 euros. Estimamos su vida útil en 3 años.

$120 euros/(12*3)meses = 4.32 euros/mes$
	
Por tanto en 6 meses:
	
$4.32 euros/mes * 6 meses = 25.92 euros brutos$

\subsubsection{Costes de materiales}
Aquí sumaremos los costes de la realización de plantillas. Estos son costes sin amortización, puesto que las plantillas una vez creadas no sirven para nada más.

\begin{itemize}
	\item Coste de las cartulinas: 10 euros.
	\item Coste de la cinta aislante: 5 euros.
\end{itemize}

Además hemos creado un soporte con Lego, con un coste aproximado de 5 euros.

\subsubsection{Costes de totales}
Vamos a estimar los costes del trabajador según los Tipos de Cotización Régimen General Ejercicio 2018\cite{cot}.

Porcentajes de cotización:
\begin{itemize}
	\item Contingencias comunes: 23.6.
	\item Desempleo: 5.5
	\item 0.6
\end{itemize}

Realizando la suma, tenemos un porcentaje de: 29.7

Por tanto el coste real del trabajador serán: $4800 + (4800*29.7) = 4942.56 euros$

Sumando el resto de costes: 
$4942.46 + 41.64 + 24.96 + 4.14 + 25.92 = 5039.12 euros$

\subsubsection{Beneficio}
En este apartado estimaremos los beneficios que se podrían obtener de este software. EL software por si solo no se puede distribuir, se ha de distribuir junto con el vehículo que tiene que guiar.

Suponemos que esta herramienta de guiado se implementa en 300 vehículos y que el coste añadido por cada vehículo que lo implemente es de 20 euros anuales.

$ 300 vehiculos * 20 euros/vehículo = 6000 euros/año$

Recuperaríamos la inversión en:

$5039.12/6000 euros anuales = 0.83 años$ 

A partir de ese tiempo, unos 10 meses, obtendríamos beneficios.


\subsection{Viabilidad legal}
En este apartado veremos las licencias del software, compatibilidades entre ellas y la licencia final de la aplicación. También, aunque no se ha trabajado con ello, veremos la legislación de robots y vehículos autónomos en España, ya que el software sí esta pensado para dirigir a un vehículo de forma autónoma.

\subsubsection{Software}
Primero veamos las licencias del software usado:
\begin{itemize}
	\item OpenCV, licencia BSD
	\item SciPy, licencia BSD
 	\item Numpy, licencia BSD
	\item MatPlotLib, licencia BSD
	\item Urllib, licencia MIT
\end{itemize}
Según la guía de licencias para python\cite{licen}, la licencia adecuada para este proyecto podría ser MIT o BSD, ya que son compatibles entre ellas, y cumplen con todas las licencias de las librerías usadas.

En este caso elegiremos licencia MIT ya que es menos restrictiva en cuanto a los avisos de Copyright, y en cuanto a publicitar del software. Con la licencia BSD se tiene que remarcar siempre que se use el software usado, quien lo creó en toda la publicidad que se haga.

La licencia Mit permite\cite{mit}:
\begin{itemize}
	\item Uso de la herramienta
	\item Distribución de la herramienta
	\item Distracción y modificación
	\item Aviso de copyrigth y garantía solo en el software que lleve partes del software marcado con esta licencia. En publicidad y otros elementos no haría falta.
	\item Mezclar la herramienta con otro software.
	\item Vender la herramienta
	\item Asignarle una licencia más restrictiva.
	\item Publicarla
\end{itemize}

La aplicación IpWebcam no la consideramos puesto que aunque ha sido usada para el desarrollo, en la versión final del producto, lo único que tenemos que suministrar al programa es una url del servidor de vídeo, por lo que el usuario podrá usar la aplicación o sistema que necesite.

\subsubsection{Legislación de robots autónomos}
Actualmente en España no hay una legislación clara sobre robots y vehículos autónomos, aun así dentro del marco Europeo ya se está trabajando en algunas normas o objetivos que en un futuro no muy lejano pueden afectar a este software\cite{ley_rob}

\begin{itemize}
	\item Necesidad de un registro de robots.
	\item Necesidad de un sistema de responsabilidad objetiva por daños causados y un sistema de seguro obligatorio.
	\item Necesidad de fijar un régimen de cotización a la seguridad social, por los dueños de los robots.
\end{itemize}

Al final estas normas son previsibles, el trabajo que está haciendo un AGV de forma autónoma lo podría hacer un trabajador que cotiza y que puede cometer fallos y generar daños. 

El registro de robots se tendrá para no tener robots "trabajando en negro", al igual que las personas.


