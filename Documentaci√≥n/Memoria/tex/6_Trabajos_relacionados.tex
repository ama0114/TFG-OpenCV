\capitulo{6}{Trabajos relacionados}
El mundo de los vehículos de guiado autónomo se remonta hasta 1953. En este año se creó lo que podemos considerar como el primer AGV de la historia, era un remolcador industrial cuya función era llevar un remolque siguiendo un cable, en un almacén de comestibles\cite{primer_agv}. 

Los sistemas de guiado han evolucionado, pasando por métodos físicos donde se usaba el magnetismo, hasta el guiado óptico. 

Siempre que se ha usado el guiado óptico ha sido enfocado hacia el suelo, tratando de ver si el vehículo se sale de la línea guía.

\imagen{gui_optico}{Sistema de guiado óptico tradicional. Fuente: goetting-agv.com}

\imagen{magnetic_guide}{Sistema de guiado por magnetismo. Fuente: roboteq.com}

De momento en el campo de la robótica industrial no se ha empleado ningún sistema de visión en perspectiva de lineas para guiar vehículos. En el campo de la automoción si que se usan este tipo de sistemas.

\section{Artículos}

Se ha encontrado un trabajo parecido, también trabajando en perspectiva, pero el sistema de guiado es mediante reconocimiento de marcas\cite{guiado_marcas}.

Sus autores son Jeisung Lee, Chang-Ho Hyun y Mignon Park. 

El sistema lo que hace es buscar una marca predefinida en el sistema, que tiene un triangulo apuntando hacia una dirección. Una vez encontrado ese triangulo, se dirige hacia él hasta estar encima, y entonces cambia su dirección hacia la dirección en la que apunta el triangulo, y así sucesivamente.

Este sistema tiene ciertas ventajas y desventajas frente al realizado en este proyecto, veamos:

\begin{itemize}
	\item La ventaja principal es el poder modificar rápidamente el trazado que ha de seguir el vehículo, simplemente moviendo las marcas a nuestro antojo. Además del ahorro económico de no tener que pintar todo el trazado que el vehículo tiene que realizar.
	
	\item Esta ventaja conlleva una desventaja, si no orientamos bien una de las marcas, se nos pierde, se tapa al poner algún objeto encima o se deteriora, podemos perder el control del vehículo, ya que seguirá recto hasta encontrar la siguiente marca.
	
\end{itemize}

\section{Proyectos}

Aunque no están dentro del ámbito de la robotica industrial, estos proyectos tienen ciertas similitudes con el proyecto realizado. La principal diferencia entre ambos es la forma de interpretación de las lineas, usan diferentes funciones, el resto es prácticamente similar.

\subsection{Resolviendo distorsión por perspectiva}

\subsubsection{Lane Line Detection using Python and OpenC}

\url{https://github.com/tatsuyah/Lane-Lines-Detection-Python-OpenCV}

\subsubsection{Advanced lane detection using computer vision}
\url{https://github.com/georgesung/advanced_lane_detection}


\subsection{Guiado mediante dos lineas en robótica industrial}

Además de estos proyectos, hay algunas pruebas en robótica industrial, pero siguiendo el mismo método que los proyectos anteriores, detección de dos lineas en lugar de una.


\subsubsection{Lane Detection TestingROS, Hydro, OpenCV}

\url{https://www.youtube.com/watch?v=dBVUjWmqICM}

\url{http://www.vision.caltech.edu/malaa/research/lane-detection/}


\subsection{Guiado mediante una línea}

Por último, destacar dos proyectos , que aunque no realizan la transformación de perspectiva, si que son capaces de guiarse con una sola línea, sin colocar la cámara exactamente sobre esta.

\subsubsection{Vision Race}

\url{https://github.com/CRM-UAM/VisionRace}

\subsubsection{AGV uI Line follower using Webcam and OpenCV (FSTG MARRAKECH)}

\url{https://www.youtube.com/watch?v=HdD6phfTOSg}